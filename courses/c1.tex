\[P(ipoteze | date) = \frac{P(date | ipoteze) \cdot P(ipoteze)}{P(date)}\]

/section{Silogisme}
/subsection{Silogisme tari}

Silogisme tari(logica Aristotelica, cu doua valori)

daca A este adevarata atunci B e adevarata:
A e adevarata => B e adevarata
B e falsa => A e falsa

/subsection{Silogisme slabe}
Silogisme slabe:
daca A e adevarata atunci B e adevarata:
B e adevarata => A devine mai plauzibil sa fie adevarata
A e falsa => B devine mai putin plauzibil sa fie adevarata

/subsection{exemplu}
A: Va incepe sa ploua pana cel tarziua la 10 dimineata
B: Cerul va deveni noros inainte de ora 10 dimineata

\section{Formalizare}
\subsection{}
A, B propozitii,\\
$\neg A$ negatia lui A\\
$AB$ sau $A \cap B$ - conjunctie\\
$A+B$ sau $A \cup B$ - disjunctie\\
$A \Rightarrow B$ - implicatie\\
$A|B$ plauzibilitate conditionala (A stiind B)\\

\subsection{Teorema lui Cox}
adevarul este reprezentat de $Pr(A|B) = 1$ si \\
falsitatea este reprezentata de $Pr(A|B) = 0$\\
$Pr(A|B) + Pr(\neg A | B) = 1$\\
$Pr(AB|C) = Pr(A|C)Pr(B|AC) = Pr(B|C)Pr(A|BC)$\\

\subsection{}

$\Omega$ = multimea tuturor rezultatelor posibilitatilor(spatiu de selectie)

$A \subseteq B => P(A) \le P(B)$\\
A, B independente => $P(A \cup B) = P(A) + P(B)$\\
$A \subseteq B => P(BA) = P(B)-P(A)$\\
$P(A|B) = \frac{P(A \cap B)}{P(B)} => P(A \cap B) = P(A|B)P(B)$\\
$P(A_1 \cap A_2 \ldots A_n) = P(A_1 | A_2 \cap A_3 \ldots A_n) \cdot \ldots \cdot P(A_1 \ldots | A_n) P(A_n)$\\


$P(A \cap B | C) = P(A|B \cap C)P(B|C)$\\

-------------------------------------------------------------\\
X - poate fi o variabila aleatoare uniforma\\\\
Daca X este o moneda masluita, atunci X este veriabila aleatoare Bernoulli\\\\
variabila alearoare binomiala: X = "nr de succese in n repetari ale unui experiment"
$P(X=k) - {C_n}^k p^k (1-p)^{n-k}$\\\\

\section{}
$F_X(v) = P(X \le v)$, daca $F_X$ continua -> X este v a continua\\
$F_X(v) = \integral_{-\infty}^v f_X(t) dt$\\
$P(a \le X \le b) = F_X(b) - F_X(a) = \integral_{a}^b f_X(t) dt$\\\\
distributia normala $f_X(t) = \frac{1}{\sqrt{2 pi} \ sigma} e^{-\frac{{t-\mu}^2}{2 \sigma^2}}$\\
distributia exponentiala $Exp(t) = f_X(t) = \lambda e^{-\lambda t}, t \gt 0$\\
